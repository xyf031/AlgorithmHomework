\documentclass{article}
\usepackage{CJK}
\usepackage{graphicx}

\begin{document}
\begin{CJK*}{GBK}{song}
\title{算法与算法复杂性理论\\ 第一次编程作业}
\author{Haha}
\maketitle

\section{算法介绍}
\qquad 算法中包含两个对点集进行操作的关键函数:getMinimal()和getCross(),它们的作用如下。

\begin{description}
\item[getMinimal():] 输入一个含n个顶点的图G以及G的一个覆盖顶点集C,如果C中没有可移除的顶点(在不破坏覆盖性的前提下移除),那么返回C。如果C中有可移除的顶点,那么对这些可移除顶点进行估计:移除该顶点后的新点集中,可移除顶点数目大小。取该数目最大的顶点,将其移除,这就是“最快速度”的含义。将移除后的新覆盖顶点集再输入进本程序递归,直到没有可移除顶点为止。这样我们得到了一个“极小”覆盖顶点集。

\item[getCross():] 输入一个含n个顶点的图G以及G的一个极小覆盖顶点集C,寻找C中是否有这样的顶点v:在与v直接相邻的顶点中,有且只有一个顶点不在C中。如果没有这样的v,那么返回G;如果有,那么任取一个,将它与那个不在C中的相邻顶点互换,再带入到getMinimal()中用最快速度缩减到极小覆盖顶点集。这样我们得到了很多的新的覆盖顶点集。

\end{description}

基于这两个函数,算法从包含所有顶点的全集开始,以“最快速度”缩减顶点集合,该循环过程中(Part 1)不断检测是否已经达到k的要求,一旦发现就输出结果并终止程序。如果这种寻找过程未能找到不大于k的覆盖点集,那么就进行第二轮的搜索(Part 2)。具体过程如下:

\begin{itemize}
\item \textbf{Part 1} For i = 1,2,...,n:\\ 1、生成$C_i=V-\{i\}$\\ 2、对$C_i$执行getMinimal()函数\\ 3、For r=1,2,...n-k重复执行getCross()函数r遍\\ 4、 将过程中得到的$C_i$都加入到覆盖点集集合covers中

\item \textbf{Part 2} 对于Part 1中得到的covers集合中的任何一对集合$C_i,C_j$而言:\\ 1、合并两个集合得到$C_{i,j}=C_i\cup C_j$\\ 2、对$C_{i,j}$执行getMinimal()函数\\ 3、For r=1,2,...n-k重复执行getCross()函数r遍

\end{itemize}

\section{运行效果}
\qquad 在作业系统中运行结果如下:

\begin{center}
\begin{figure}[ht]
\scalebox{0.23}{\includegraphics[]{TestResult.png}}
\end{figure}
\end{center}


\end{CJK*}
\end{document}
