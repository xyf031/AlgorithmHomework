\documentclass{article}
\usepackage{CJK}
\usepackage{graphicx}

\begin{document}
\begin{CJK*}{GBK}{song}
\title{算法与算法复杂性理论\\ 第二次编程作业:删数}
\maketitle

\section{算法介绍}
\qquad 将这个n位数记为$a_1a_2...a_n$,现在要删掉其中k位,同时得到最大的结果。

首先考虑k=1的情形。删掉1位之后的数是n-1位,从最高位依次向下考虑:最高位一定是$a_1,a_2$两个中的一个,因此如果$a_1<a_2$那么一定要删除$a_1$才可以得到可能的最大值,此时删数完成;如果$a_1>=a_2$那么不能删除$a_1,a_2$,所以要考虑第二个最高位,而这个位置的情形和最高位是一模一样的,一直循环下去。如果到最后一位依旧没有出现$a_i<a_{i+1}$的情况,例如54321 或者22222,那么就删掉最后一位。由于数位有限,所以程序必然结束。

有了k=1的基础,可以考虑$k>1$的情形。这是一个符合贪心算法要求的过程,任何一个删k个数的方案都可以对删除的数字进行从高位到低位的排序,其中最靠近高位的那个数字一定和k=1的删除结果是一样的,否则不可能得到最大值;第二个最靠近高位的数字也符合类似规律。所以程序可以反复执行k=1的删数过程,并将删数之后的结果作为输入传到下一轮的程序中,一共执行k轮。



\section{程序结构}
\qquad 将数字直接存储为长度为100000的数组(输入数据中没有n的大小,所以直接用n的最大值分配空间),用-2标记数字的结束,用-1标记某位被删除。

int getNext(int data[], int thisOne)函数用来返回第thisOne元素的下一个未被删除元素的值,如果thisOne元素是末尾元素,那么返回10,这样一定可以在deleteNum函数中把thisOne删掉,也就是删除末尾。

void deleteNum(int data[], int index)函数用来判断第index位是否需要删除,若需要则把元素改记为-1,删除的条件就是该元素比下一个未被删除元素小。

void solution(int \&K, int data[])函数用来读取一个n位整数以及k,之后调用k次deleteNum函数。并将结果打印出来。



\section{实验结果}
\qquad 在作业系统中运行结果如下:

\begin{center}
\begin{figure}[ht]
\scalebox{0.28}{\includegraphics[]{Result.png}}
\end{figure}
\end{center}


\end{CJK*}
\end{document}
